\documentclass[margin,line,10pt]{res}
\usepackage{textgreek}
\usepackage{hyperref}

\oddsidemargin -.5in
\evensidemargin -.5in
\textwidth=6.0in
\itemsep=0in
\parsep=0in
% if using pdflatex:
\setlength{\pdfpagewidth}{\paperwidth}
\setlength{\pdfpageheight}{\paperheight}

\newenvironment{list1}{
  \begin{list}{\ding{113}}{%
      \setlength{\itemsep}{0in}
      \setlength{\parsep}{0in} \setlength{\parskip}{0in}
      \setlength{\topsep}{0in} \setlength{\partopsep}{0in}
      \setlength{\leftmargin}{0.17in}}}{\end{list}}
\newenvironment{list2}{
  \begin{list}{$\bullet$}{%
      \setlength{\itemsep}{0in}
      \setlength{\parsep}{0in} \setlength{\parskip}{0in}
      \setlength{\topsep}{0in} \setlength{\partopsep}{0in}
      \setlength{\leftmargin}{0.2in}}}{\end{list}}


\begin{document}

\name{Dr. Alexandra K. Diem, PhD \vspace*{.1in}}

\address{alexandra.diem@gmail.com, \url{https://www.akdiem.com}, \url{https://github.com/akdiem}}

\begin{resume}

  {\sc Research Interests} Cardiovascular modelling, biological flows, cerebral fluid dynamics, nutrient dynamics, research software development\\

I develop computational models of biological flows across various scales based on porous media. I obtained my PhD for the development of computational models of waste clearance from the brain to discover its mechanism of failure as a precursor for Alzheimer's disease. My work demonstrated that the popular hypothesis based on arterial pulsations does not hold, which was confirmed experimentally. I further provided computational evidence for a new hypothesis linking waste clearance to mental and physical stimulation. My current work based on the same numerical methods, focusses on the development of a cardiac perfusion model covering across spatial and time scales. The model is motivated by the many open questions related to adverse effects of ventricular remodelling that lead to heart failure, as well as the advancement of novel treatment methods via the targeted delivery of therapy-loaded nanoparticles.\\

\vspace*{-.2in}

{\sc Current Position}\\
{\bf Postdoctoral Fellow}, Department of Computational Physiology, Simula Research Laboratory, Oslo, Norway\\
{\em Project:} Cardio Ultraefficient nanoParticles for Inhalation of Drug prOducts (CUPIDO), H2020 Project 720834\\

\vspace*{-.2in}

{\sc Previous Positions}\\
\vspace*{-.35in}
\section{\sc 01/2017--11/2017}{\bf Doctoral Prize Research Fellow}, Computational Engineering and Design, Faculty of Engineering and the Environment, University of Southampton, UK\\
Individual fellowship funded by the EPSRC.\\

\vspace*{-.2in}

{\sc Education}\\
\vspace*{-.35in}
\section{\sc 2016}{\bf PhD}, Faculty of Engineering and the Environment, University of Southampton, UK\\
Supervisors: Prof. Neil W. Bressloff, Prof. Roxana O. Carare, Dr. Giles Richardson\\
\vspace*{-.35in}
\section{\sc 2012}{\bf Diplom Bioinformatik} (equivalent to MSc), Fakultät für Mathematik und Informatik, Friedrich-Schiller-Universität Jena, Germany\\
Supervisor: Dr. Peter Dittrich\\

\vspace*{-.2in}

{\sc Research Mobility}\\
\vspace*{-.35in}
\section{\sc 01/2019--02/2019}Cardiac Mechanics Research Group, University of California San Diego, California, USA\\
Development of a poroelastic heart mechanics solver in collaboration with Prof. Andrew McCulloch
\vspace*{-.35in}
\section{\sc 07/2015--08/2015}Mechanobiology Lab, Division of Biomedical Engineering, University of Cape Town, South Africa\\
Research visit as part of my PhD project and funded by a Worldwide Universities Network Research Mobility grant under the supervision of Dr. Malebogo Ngoepe\\

\vspace*{-.2in}

{\sc Fellowships, Awards and Prizes}\\
\vspace*{-.35in}
\section{\sc 2017}{\bf Doctoral Prize}, EPSRC Fellowship grant, Faculty of Engineering and the Environment, University of Southampton, UK\\
\vspace*{-.35in}
\section{\sc 2016}{\bf Take Off}, entrepreneurial competition award, Future Worlds startup accelerator, University of Southampton, UK\\
\vspace*{-.35in}
\section{\sc 2016}{\bf Three Minute Thesis Faculty Runner Up}, presentation competition, Faculty of Engineering and the Environment, University of Southampton, UK\\
\vspace*{-.35in}
\section{\sc 2015}{\bf Research Mobility Programme}, Worldwide Universities Network research visit grant, GBP 3,100.\\
\vspace*{-.35in}
\section{\sc 2010}{\bf Dean's Commendation for High Achievement}, academic merit award, University of Queensland, Australia\\
\vspace*{-.35in}
\section{\sc 2007--2012}{\bf Academic Scholarship}, Konrad Adenauer Association, Germany.\\

\vspace*{-.2in}

{\sc Supervision}\\
\vspace*{-.35in}
\section{\sc 2019}{\bf 1 MSc student}, Department of Computational Physiology, Simula Research Laboratory, Norway\\
Numerical comparison between a pure porous and poroelastic approach to modelling myocardial tissue.\\
\vspace*{-.35in}
\section{\sc 2018}{\bf 1 summer intern}, Department of Computational Physiology, Simula Research Laboratory, Norway\\
Implementation of 1D blood flow equations using the finite element framework FeniCS in Python. The internship resulted in a journal publication in the Journal of Open Source Software.\\

\vspace*{-.2in}

{\sc Teaching}\\
\vspace*{-.35in}
\section{\sc 10/2013--12/2016}{\bf Demonstrator}, University of Southampton, UK\\
Supervision of Python programming labs, sheet metal and metrology workshop labs for 1st year Engineering students\\
\vspace*{-.35in}
\section{\sc 10/2011--07/2012}{\bf Demonstrator}, Friedrich Schiller University of Jena, Germany\\
Independent planning and running of tutorial sessions for lectures on Evolutionary Algorithms and Bio Systems Analysis.\\

\vspace*{-.2in}

{\sc Organisation of Scientific Meetings}\\
\vspace*{-.35in}
\section{\sc 08/2014}{\bf Conference Chair}, 4th Student Conference on Complexity Science, Brighton, UK\\

\vspace*{-.2in}

{\sc Publications}\\
h-index: 5, i10-index: 5\\
\vspace*{-.35in}
\section{\sc 2018}
{\bf Diem AK}, Carare RO, Bressloff NW (2018) A control mechanism for intramural periarterial draiange via astrocytes: How neuronal activity could improve waste clearance from the brain. \textit{Plos One} 13:10, e0205276, \url{https://doi.org/10.1371/journal.pone.0205276}\\
Agdestein S, Valen-Sendstad K, {\bf Diem AK} (2018) Artery.FE: An Implementation of the 1D blood flow equations in FEniCS. \textit{The Journal of Open Source Software} 3, 1107, \url{https://doi.org/10.21105/joss.01107}\\        
\vspace*{-.35in}
\section{\sc 2017}
Rougier, NP, Hinsen K, Alexandre F, Arildsen T, Barba LA, et al. (2017) Sustainable computational science: the ReScience initiative \textit{PeerJ Computer Science} 3, e142, \url{https://doi.org/10.7717/peerj-cs.142}, citations: 14\\
{\bf Diem AK}, Bressloff NW, Carare RO, MacGregor Sharp M, Richardson G (2017) Arterial pulsations cannot drive intramural periarterial drainage: Significance for Alzheimer's disease. \textit{Frontiers in Neuroscience} 11:475, \url{https://doi.org/10.3389/fnins.2017.00475}, citations: 11 (10)\\
{\bf Diem AK}, Bressloff NW (2017) VaMpy: A Python Package to Solve 1D Blood Flow Problems. {\em Journal of Open Research Software} 5:17, \url{http://doi.org/10.5334/jors.159}, citations: 3 (1)\\
{\bf Diem AK} (2017) Chemical Signalling in the Neurovascular Unit. \textit{ReScience} 3(1): 9, \url{https://github.com/akdiem/ReScience-submission/tree/Diem-2017}\\
\vspace*{-.35in}
\section{\sc 2016}
Sharp MK, {\bf Diem AK}, Weller RO, Carare RO (2016) Peristalsis with Oscillating Flow Resistance: A Mechanism for Periarterial Clearance of Amyloid Beta from the Brain. \textit{Annals of Biomedical Engineering} 44(5): 1553--1565, \url{https://dx.doi.org/10.1007/s10439-015-1457-6}, citations: 17 (15)\\
{\bf Diem AK}, Tan M, Bressloff NW, Hawkes C, Morris AWJ, et al. (2016) A Simulation Model of Periarterial Clearance of Amyloid-\textbeta from the Brain. \textit{Frontiers in Aging Neuroscience} 8:18, \url{https://doi.org/10.3389/fnagi.2016.00018}, citations: 16 (16)\\

\end{resume}
\end{document}
