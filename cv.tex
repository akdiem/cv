\documentclass[margin,line,10pt]{res}
\usepackage{textgreek}
\usepackage{hyperref}

\oddsidemargin -.5in
\evensidemargin -.5in
\textwidth=6.0in
\itemsep=0in
\parsep=0in
% if using pdflatex:
\setlength{\pdfpagewidth}{\paperwidth}
\setlength{\pdfpageheight}{\paperheight} 

\newenvironment{list1}{
  \begin{list}{\ding{113}}{%
      \setlength{\itemsep}{0in}
      \setlength{\parsep}{0in} \setlength{\parskip}{0in}
      \setlength{\topsep}{0in} \setlength{\partopsep}{0in} 
      \setlength{\leftmargin}{0.17in}}}{\end{list}}
\newenvironment{list2}{
  \begin{list}{$\bullet$}{%
      \setlength{\itemsep}{0in}
      \setlength{\parsep}{0in} \setlength{\parskip}{0in}
      \setlength{\topsep}{0in} \setlength{\partopsep}{0in} 
      \setlength{\leftmargin}{0.2in}}}{\end{list}}


\begin{document}

\name{Dr. Alexandra K. Diem, PhD \vspace*{.1in}}

\address{alexandra.diem@gmail.com, \url{https://www.akdiem.com}, \url{https://github.com/akdiem}}

\begin{resume}

{\sc Research Interests} Cardiovascular modelling, biological flows, cerebral fluid dynamics, nutrient dynamics, research software development\\

My research focusses on developing coupled and multi-scale analytical and numerical models of intramural periarterial drainage of waste products from the brain. My major contributions to the field are disproving the often cited hypothesis that periarterial drainage through the cerebrovascular basement membrane is driven by arterial pulsations and providing the first evidence for a new hypothesis, which links waste clearance to functional hyperaemia. This new hypothesis lines up with recommendations on the reduction of one's individual risk of developing Alzheimer's disease. I am passionate about sustainable research software development and have thus published software and data from my research projects on Github, as well as in research software journals.\\ 
  
\vspace*{-.2in}

{\sc Current Position}\\
{\bf Postdoctoral Fellow}, Department of Computational Physiology, Simula Research Laboratory, Oslo, Norway\\

\vspace*{-.2in}

{\sc Academic Degrees}\\
\vspace*{-.35in}
\section{\sc 2016}{\bf PhD}, Faculty of Engineering and the Environment, University of Southampton, UK\\
Supervisors: Prof. Neil W. Bressloff, Prof. Roxana O. Carare, Dr. Giles Richardson\\
\vspace*{-.35in}
\section{\sc 2012}{\bf Diplom Bioinformatics} (equivalent to MSc), Faculty of Mathematics and Computer Science, Friedrich Schiller University of Jena, Germany\\
Supervisors: Dr. Peter Dittrich\\

\vspace*{-.2in}

{\sc Research Experience}\\
\vspace*{-.35in}
\section{\sc 12/2017--present}{\bf Postdoctoral Fellow}, Department of Computational Physiology, Simula Research Laboratory, Oslo, Norway\\
{\em Project:} Cardio Ultraefficient nanoParticles for Inhalation of Drug prOducts (CUPIDO), H2020 Project 720834\\
\vspace*{-.35in}
\section{\sc 01/2017--11/2017}{\bf Doctoral Prize Research Fellow}, Computational Engineering and Design, Faculty of Engineering and the Environment, University of Southampton, UK\\
Individual fellowship funded by the EPSRC.\\
\vspace*{-.35in}
\section{\sc 07/2015--08/2015}{\bf Visiting Researcher}, Mechanobiology Lab, Division of Biomedical Engineering, University of Cape Town, South Africa\\
Research visit as part of my PhD project and funded by a Worldwide Universities Network Research Mobility grant under the supervision of Dr. Malebogo Ngoepe\\
\vspace*{-.35in}
\section{\sc 10/2012--11/2016}{\bf PhD Student}, Institute for Complex Systems Simulation \& Computational Engineering and Design, Faculty of Engineering and the Environment, University of Southampton, UK\\
as part of the Complex Systems Simulation Doctoral Training Centre 3+1 PhD programme\\
{\em Thesis:} The Role of Arterial Pulsations in Perivascular Drainage of A\textbeta: Implications for Alzheimer's Disease\\    
\vspace*{-.35in}        
\section{\sc 04/2012--09/2012}{\bf Research Assistant}, Bio Systems Analysis Group, Faculty of Mathematics and Computer Science, Friedrich Schiller University of Jena, Germany\\
{\em Project:} Artificial Wet Neuronal Networks from Compartmentalised Excitable Chemical Media (NEUNEU), FP7 Project 248992\\
Responsibility for the implementation of deliverable 4.5: Software models of sorting and classification implemented in acyclic networks of chemical droplets\\
\vspace*{-.35in}
\section{\sc 11/2010--04/2012}{\bf Research Student}, Bio Systems Analysis Group, Faculty of Mathematics and Computer Science, Friedrich Schiller University of Jena, Germany\\
{\em Project:} Artificial Wet Neuronal Networks from Compartmentalised Excitable Chemical Media (NEUNEU), FP7 Project 248992\\
Design and concept development of deliverable 4.5: Software models of sorting and classification implemented in acyclic networks, assistance with software development tasks and at project meetings\\
Tutor for lectures on Evolutionary Algorithms and Bio Systems Analysis\\
\vspace*{-.35in}
\section{\sc 02/2010--06/2010}{\bf Research Student}, Computational Genomics Group, Institute for Molecular Bioscience, University of Queensland, Australia\\
Implementation of file format parsing software for automated protein-protein interaction detection\\
as part of a study abroad semester, final GPA 6.25 (equivalent to UK 1st class)\\
\vspace*{-.35in}
\section{\sc 10/2007--03/1012}{\bf Diplom Bioinformatics} (equivalent to MSc), Grade 1.2 (equivalent to UK 1st class), Friedrich Schiller University of Jena, Germany\\ 

\vspace*{-.2in}

{\sc Teaching}\\
\vspace*{-.35in}
\section{\sc 10/2013--12/2016}{\bf Demonstrator}, University of Southampton, UK\\
Supervision of Python programming labs, sheet metal and metrology workshop labs for 1st year Engineering students\\
\vspace*{-.35in}
\section{\sc 10/2011--07/2012}{\bf Demonstrator}, Friedrich Schiller University of Jena, Germany\\
Independent planning and running of tutorial sessions for lectures on Evolutionary Algorithms and Bio Systems Analysis.\\

\vspace*{-.2in}

{\sc Other Academic Activities}\\
\vspace*{-.35in}
\section{\sc 10/2017}{\bf Keynote Speaker}, Workshop ``Biomechanics of living systems: from cells to organisms'' at the University of Oslo, Norway\\
\vspace*{-.35in}
\section{\sc 08/2014}{\bf Conference Chair}, 4th Student Conference on Complexity Science, Brighton, UK\\

\vspace*{-.2in}

{\sc Grants, Honours and Awards}\\
\vspace*{-.35in}
\section{\sc 03/2016}{\bf Doctoral Prize}, EPSRC Fellowship grant, GBP 29,301.\\
\vspace*{-.35in}
\section{\sc 03/2015}{\bf Research Mobility Programme}, Worldwide Universities Network research visit grant, GBP 3,100.\\        
\vspace*{-.35in}
\section{\sc 06/2010}{\bf Dean's Commendation for High Achievement}, Academic merit award, University of Queensland, Australia\\
\vspace*{-.35in}
\section{\sc 10/2007--03/2012}{\bf Scholarship}, Konrad Adenauer Association, Germany.\\        

\vspace*{-.2in}

{\sc Relevant Publications}\\
\vspace*{-.35in}
\section{\sc 2017}

{\bf Diem AK}, Carare RO, Bressloff NW (2017) A control mechanism for intramural periarterial draiange via astrocytes: How neuronal activity could improve waste clearance from the brain, in preparation, \url{http://arxiv.org/abs/1710.01117}\\
{\bf Diem AK}, Bressloff NW, Carare RO, MacGregor Sharp M, Richardson G (2017) Arterial pulsations cannot drive intramural periarterial drainage: Significance for Alzheimer's disease. \textit{Frontiers in Neuroscience} 11:475, \url{https://doi.org/10.3389/fnins.2017.00475}\\
{\bf Diem AK}, Bressloff NW (2017) VaMpy: A Python Package to Solve 1D Blood Flow Problems. {\em Journal of Open Research Software} 5:17, \url{http://doi.org/10.5334/jors.159}\\
{\bf Diem AK} (2017) Chemical Signalling in the Neurovascular Unit. \textit{ReScience} 3(1): 9, \url{https://github.com/akdiem/ReScience-submission/tree/Diem-2017}\\
\vspace*{-.35in}
\section{\sc 2016}
{\bf Diem AK}, Tan M, Bressloff NW, Hawkes C, Morris AWJ, Weller RO, Carare RO (2016) A Simulation Model of Periarterial Clearance of Amyloid-\textbeta from the Brain. \textit{Frontiers in Aging Neuroscience} 8:18, \url{https://doi.org/10.3389/fnagi.2016.00018}\\
Sharp MK, {\bf Diem AK}, Weller RO, Carare RO (2016) Peristalsis with Oscillating Flow Resistance: A Mechanism for Periarterial Clearance of Amyloid Beta from the Brain. \textit{Annals of Biomedical Engineering} 44(5): 1553--1565, \url{https://dx.doi.org/10.1007/s10439-015-1457-6}\\

\end{resume}
\end{document}
